\documentclass[12pt,a4paper]{article}
% \usepackage[english]{babel}
% \usepackage[utf8x]{inputenc}

\usepackage{graphicx} % Required for inserting images.
\usepackage[margin=25mm]{geometry}
\parskip 4.2pt  % Sets spacing between paragraphs.
% \renewcommand{\baselinestretch}{1.5}  % Uncomment for 1.5 spacing between lines.
\parindent 8.4pt  % Sets leading space for paragraphs.
\usepackage[font=sf]{caption} % Changes font of captions.

\usepackage{amsmath}
\usepackage{amsfonts}
\usepackage{amssymb}
\usepackage{siunitx}
\usepackage{verbatim}
\usepackage{hyperref} % Required for inserting clickable links.
\usepackage{natbib} % Required for APA-style citations.


\title{President Eletction Application Documentation}
\author{Martynas Galkinas}

\begin{document}
\maketitle

\begin{abstract}
    200 words, and then we're done with the abstract.
\end{abstract}

\section{Introduction}\label{sec:intro}
You have got to check out Fig.~\ref{fig:venn}: it is amazing.

Here is an example citation when you want an author name like \cite{collins2011a} to appear in the text. And here's how to do a parenthetic citation, when you want to mention a reference at the end of a sentence or part of a sentence \citep{collins2013}.

It is possible to cite multiple references at the same time \citep{collins2011b,collins2016,lunn2007a,lunn2007b,ross2006,shannon1948}.

\section{Project}
\subsection{Requirement Specification}
Scope of the project: h and loc
Conditions:
\begin{itemize}
    \item
        There are multple candidates
    \item
        Voter can vote for only one of them
    \item
        No voter can change his/her decision once submitted
    \item
        There is no need to keep track of any voter data except and identifier,
        region and their vote
    \item
        Assume that customer is identified by third party service, do not
        implement registration or authentication.
\end{itemize}


Tasks:
\begin{enumerate}
    \item
        Implement an endpoint that returns a list of candidates available:
        \begin{itemize}
            \item name
            \item number on the list
            \item short summary of their agenda
        \end{itemize}
    \item
        Implement an endpoint that enables voting for the candidate
    \item
        Implement endpoints that return voting result reports
        \begin{itemize}
            \item Overall distribution of votes amongst candidates
            \item Voting result distribution amongst different regions
            \item The winner endpoint. It must return a single candidate
                if he/she was voted for by more that 50\%. Otherwise
                it must return two most voted candidates.
        \end{itemize}
\end{enumerate}

Requirements:
\begin{itemize}
    \item
        You can use any of the followwing programming languages Java, Scala, Kotlin, Groovy
    \item
        You may use any frameworks you need.
    \item
        There are no specific requirements for data storage. You can keep it in
        memory
    \item
        All interaction with an application must be implemented either as REST
        or GraphQL endpoints
    \item
        Be mindful about naming and comments. Your core must be readable and
        clean.
    \item
        Your final delivery must be either Maven or Gradle Project
\end{itemize}

\subsubsection{System functions}
Usage diagram
Action diagram

\subsubsection{System project}
dba diagram
class diagram
Internal diagram
Installation diagram
API documentation

\section{Testing}
\subsection{Integration tests}
\subsection{Unit tests}

\section{User Documentation}

Did you see what I wrote in Sec.~\ref{sec:intro}? Not too controversial I hope!

\begin{figure}[htbp!]
    \begin{center}
        \includegraphics[width=0.5\columnwidth]{venn_discoveries.pdf}
    \end{center}
    \caption{Yes, put a few words or sentences here explaining what is in the figure.}
    \label{fig:venn}
\end{figure}

\bibliographystyle{apalike}
\bibliography{example}

\end{document}
